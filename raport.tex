%%%%%%%%%%%%%%%%%%%%%%%%%%%%%%%%%%%%%%%%%%%
%   Simple and elegant academic report    %
%   Copyright by Artur M. Brodzki, 2019   %
%%%%%%%%%%%%%%%%%%%%%%%%%%%%%%%%%%%%%%%%%%%

\documentclass{eiti-raport}

\usepackage[
	english,
	polish
]{babel}
\usepackage{polski}

%------------------------------------------

\begin{document}

\author{Artur M. Brodzki}
\date{\today}
\subject{BCYB 2019L}
\title{Analiza raportu ,,Hammertoss'' \\ grupy Fire Eye}

\pagenumbering{arabic}
\maketitle

%------------------------------------------
% MAIN CONTENTS
%------------------------------------------

\begin{multicols*}{2}

\section{Wstęp} \label{sec:intro}
W niniejszym dokumencie przeanalizowano raport grupy FireEye dot. złośliwego oprogramowania typu APT, tzw. ,,Hammertoss'', wykorzystującego zaawansowane metody steganografii i unikania wykrycia. Analizę przeprowadzono z punktu widzenia koncepcji Intrusion Kill Chain oraz matrycy MITRE Attack. Rozważono również potencjalne sposoby obrony przed tego rodzaju zagrożeniami. 

\section{Analiza IKC} \label{sec:2}
Według koncepcji IKC (ang. \textit{Intrusion Kill Chain}), atak sieciowy sklada się z 7 podstawowych faz:
\begin{enumerate}
	\itemsep0em
	\item Rekonesans,
	\item Uzbrajanie,
	\item Dostarczenie,
	\item Eksploitacja,
	\item Instalacja,
	\item Command and Control,
	\item Akcje atakujących.
\end{enumerate}
Przedmiotowy raport nie skupia się na pierwszych pięciu fazach ataku sieciowego, począwszy od rekonesansu do instalacji; zamiast tego, otrzymujemy szczegółowey opis działania backdoora już po jego pomyślnym dostarczeniu i zainstalowaniu na maszynie ofiary. ,,Hammertoss'' stosuje zaawansowane techniki steganograficzne, wykorzystując media społecznościowe do ukrytego przekazywania komend od operatora backdoora -- jest to faza Command and Control. Dodatkowo, backdoor posiada możliwość wykradania danych z zainfekowanej maszyny i przekazywania ich na zdalny serwer atakującego. Odbywa się to poprzez upload danych na konto w chmurze, którego dane zostały przekazane w kanale C\&C. Jest to faza akcji atakujących. 

\section{MITRE Attack}
Matryca MITRE Attack stanowi metodę klasyfikacji ataków sieciowych w sposób dwupoziomowy, z podziałem na taktyki -- działania służące do osiągnięcia określonego celu oraz techniki -- w ramach zadanej taktyki, konkretny sposób (zachowanie) pozwalające na osiągnięcie zadanego celu.

W przedmiotowym raporcie zawarto opisy trzech głównych taktyk realizowanych przez programowanie ,,Hammertoss'': \textit{defense evasion} (unikanie obrony), \textit{exfiltration} (infiltracja\footnote{W języku angielskim występują osobne określenia \textit{exfiltration} oznaczające ukryte wynoszenie danych oraz \textit{infiltration} posiadające głównie znaczenie hydrologiczne, ale też oznaczające uzyskanie nieautoryzowanego dostępu do miejsca lub zasobów. W języku polskim słowo ,,eksfiltracja'' nie występuje i jako takie stanowi niepoprawną kalkę językową, a zarówno nieautoryzowany dostęp do danych jak i ich wynoszenie na zewnątrz mieszczą się w zakresie słowa ,,infiltracja'', używanego głównie w żargonie wojskowym.}) oraz \textit{command and control}:
\begin{enumerate}
	\item \textit{Defense evasion}: ,,Hammertoss'' wykorzystuje technikę \textit{Deobfuscate/Decode Files or Information}, a mianowicie ukrywanie danych w plikach obrazkowych, zamieszczanych na generowanych pseudolosowo kontach twitterowych sprawiających wrażenie kont zwyczajnych użytkowników;
	\item \textit{Exfiltration}: ,,Hammertoss'' oprócz tego, że ukrywa dane (steganografia) to stosuje jeszcze szyfrowanie (\textit{Data encryption}), wyprowadzając dane poprzez upload na zadane konto w chmurze; dane tego konta przekazywane sa w kanale C\&C, jest to technika \textit{Exfiltration Over Alternative Protocol};
	\item \textit{Command and control}: do komunikacji z operatorem stworzono autorski protokół komunikacji (\textit{Custom Command and Control Protocol}), którzy wykorzystuje Twittera jako warstwę pośrednią (\textit{Connection Proxy}), ukrywający komendy do przekazania w plikach obrazkowych (\textit{Data Obfuscation}).
\end{enumerate}
Techniki te połączone razem tworzą narzędzie, które jest szczególnie trudne do wykrycia. 

\section{Techniki obrony}
,,Hammertoss'' stosuje znany algorytm ukrywania danych w pliku obrazkowym, możliwe jest więc stosowanie narzędzi przeznaczonych do wykrywania tego rodzaju ukrytej transmisji za pomocą sygnatur. Wydaje się jednak, że w wypadku nieznanego zagrożenia realizującego działania steganograficzne, główną linią obrony powinna być analiza ruchu przez systemy typu IDS/IPS. Wykrycie anomalii w ruchu sieciowym, np. ponadprzeciętnie częstych zapytań do Twittera, powinno zwrócić uwagę administratorów na możliwość zainfekowania sieci przez nieznany malware, wykorzystujący Twittera jako medium komunikacji.

\section{Podsumowanie} \label{sec:summary}
Raport FireEye stanowi interesującą analizę niebezpiecznego oprogramowania, które dzięki połączeniu wielu różnych technik w jednym narzędziu stanowi duże wyzwanie dla cyberbezpieczeństwa. 


\end{multicols*}
\end{document}
